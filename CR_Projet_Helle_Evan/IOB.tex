\section{IO Buffer}

\subsection{Fonctionnement}

Un IO Buffer est un composant matériel utilisé dans les circuits pour gérer les signaux d'entrée et de sortie entre le circuit et le monde extérieur.\\
L'implémentation de L'IO Buffer que nous allons utilisé nous nous a été fourni.\\

Voici la représentation schématique de l'IO Buffer que nous allons utilisé dans notre projet :\\
% - Image - %
\begin{figure}[H]
\centering
\includegraphics[width=0.5\textwidth]{Image/IOB.jpeg}
\caption{IO buffer}
\end{figure}

L'IO Buffer est composé de plusieurs entrées et sorties :\\
\begin{itemize}
    \item \textbf{I} : Donnée d'entrée (input data)
    \item \textbf{O} : Donnée de sortie (output data)
    \item \textbf{T} : Trigger, permet de contrôler le mode de l'IO Buffer (entrée ou sortie qui servira à gerer l'ecriture ou la lecture dans notre cas)
    \item \textbf{IO} : Bus de données bidirectionnel (data bus)
\end{itemize}

\subsection{Test Bench}

L'objectif de faire un test bench est de vérifier le bon fonctionnement de l'IO Buffer avec la SRAM.\\
De plus cela premettra de de comprendre son fonctionnement avant de l'intégrer dans le projet final.\\
Il est important de faire ce test car il est nécessaire d'avoir un décalage entre la donnée et les signaux de contrôle au niveau de la SRAM.\\

% - Image - %
\begin{figure}[H]
\centering
\includegraphics[width=0.7\textwidth]{Image/decalage_data.png}
\caption{Extrait ennoncé projet}
\end{figure}

D'après l'ennoncé du projet et la datasheet de la SRAM, il est nécessaire d'avoir un décalage d'un cycle d'horloge entre la donnée et les signaux de contrôle.\\

\subsubsection{Code}


\subsubsection{Simulation}


% - Image - %
\begin{figure}[H]
\centering
\includegraphics[width=0.7\textwidth]{Image/sim_IOB.png}
\caption{Simulation IO Buffer}
\end{figure}

% - Image - %
\begin{figure}[H]
\centering
\includegraphics[width=0.7\textwidth]{Image/sim_IOB_ZZZ.png}
\caption{Haute impédance à la lecture de la donnée}
\end{figure}