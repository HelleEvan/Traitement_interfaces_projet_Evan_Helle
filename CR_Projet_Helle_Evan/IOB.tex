\section{IO Buffer}

\subsection{Fonctionnement}

Un IO Buffer est un composant matériel utilisé dans les circuits pour gérer les signaux d'entrée et de sortie entre le circuit et le monde extérieur.\\
L'implémentation de L'IO Buffer que nous allons utilisé nous nous a été fourni.\\

Voici la représentation schématique de l'IO Buffer que nous allons utilisé dans notre projet :\\
% - Image - %
\begin{figure}[H]
\centering
\includegraphics[width=0.5\textwidth]{Image/IOB.jpeg}
\caption{IO buffer}
\end{figure}

L'IO Buffer est composé de plusieurs entrées et sorties :\\
\begin{itemize}
    \item \textbf{I} : Donnée d'entrée (input data)
    \item \textbf{O} : Donnée de sortie (output data)
    \item \textbf{T} : Trigger, permet de contrôler le mode de l'IO Buffer (entrée ou sortie qui servira à gerer l'ecriture ou la lecture dans notre cas)
    \item \textbf{IO} : Bus de données bidirectionnel (data bus)
\end{itemize}

\subsection{Test Bench}

L'objectif de faire un test bench est de vérifier le bon fonctionnement de l'IO Buffer avec la SRAM.\\
De plus cela permettra de de comprendre son fonctionnement avant de l'intégrer dans le projet final.\\
Il est important de faire ce test car il est nécessaire d'avoir un décalage entre la donnée et les signaux de contrôle au niveau de la SRAM.\\

% - Image - %
\begin{figure}[H]
\centering
\includegraphics[width=0.7\textwidth]{Image/decalage_data.png}
\caption{Extrait énoncé projet}
\end{figure}

D'après l'énoncé du projet et la datasheet de la SRAM, il est nécessaire d'avoir un décalage d'un cycle d'horloge entre la donnée et les signaux de contrôle.\\

\subsubsection{Code}

Pour le code du test bench, nous sommes partie sur une ecriture de donnée à l'adresse 1 et 2, puis une lecture de ces deux adresses.\\

% - Code source avec minted - %
\begin{minted}[frame=lines,
    framesep=2mm,
    baselinestretch=1.2,
    bgcolor = lightgray,
    fontsize=\footnotesize,
    linenos
    ]{VHDL}
tb : PROCESS
BEGIN
-- init
nCKE    <= '0';
nADVLD  <= '0';
nOE     <= '0';-- output enable
nCE     <= '0';
nCE2    <= '0';
CE2     <= '1';
SA      <= (others => '0');
Trig    <='1'; -- se mettre en "lecture" le temps de l'init pour ne pas ecrire n'importe quoi 

wait for 6*(TCLKL+TCLKH);
SA 		<= "000"&x"0001";
Trig    <= '0'; -- ecriture à l'adresse 1
wait for 1*(TCLKL); -- pour travailler sur front montant
ENTREE  <= (others => '1'); -- decalage de la donnée d'un cycle 
                            --par rapport à l'adresse et la commande
 
   
wait for 2*(TCLKL+TCLKH);
SA 		<= "000"&x"0002"; -- eriture à l'adress 2
wait for 1*(TCLKL+TCLKH);
ENTREE  <= ENTREE + 1;-- decalage de la donnée d'un cycle par
                      -- rapport à l'adresse et la commande
    
wait for 2*(TCLKH+TCLKL);
SA 		<= "000"&x"0001"; -- lecture à l'adresse 1
Trig <= '1';
wait for 2*(TCLKH+TCLKL);
SA 		<= "000"&x"0002"; -- lecture à l'adresse 2

wait; -- will wait forever
END PROCESS;
\end{minted}

Ici la donnée est bien décalée d'un cycle par rapport à l'adresse et la commande.\\
Quand le trigger est à 0, on est en écriture, et quand il est à 1, on est en lecture. C'est le cas pour la SRAM et l'IO buffer. Donc nous avons connecter par un fil ces deux I/O.\\

% - Code source avec minted - %
\begin{minted}[frame=lines,
    framesep=2mm,
    baselinestretch=1.2,
    bgcolor = lightgray,
    fontsize=\footnotesize,
    linenos
    ]{VHDL}
SRAM1 : mt55l512y36f port map
(DQ, SA, '0', CLKO_SRAM, nCKE, nADVLD, '0',
'0', '0', '0', Trig, nOE, nCE, nCE2, CE2, '0');

IOb: for I in 0 to 35 generate
Iobx: IOBUF_F_16  port map(
O => SORTIE(I),
IO => DQ(I),  
I => ENTREE(I), 
T => Trig
);
end generate;
\end{minted}

\subsubsection{Simulation}

Voici le résultat de la simulation :\\
% - Image - %
\begin{figure}[H]
\centering
\includegraphics[width=1\textwidth]{Image/sim_IOB.png}
\caption{Simulation IO Buffer}
\end{figure}
Les données sont bien écrites aux bonnes adresses (1 et 2) et lues correctement par la suite.\\
Au passage à la lecture, le bus bidirectionnel passe en haute impédance, ce qui est le comportement attendu.\\
Voici un zoom sur le passage en haute impédance :\\

% - Image - %
\begin{figure}[H]
\centering
\includegraphics[width=1\textwidth]{Image/sim_IOB_ZZZ.png}
\caption{Haute impédance à la lecture de la donnée}
\end{figure}